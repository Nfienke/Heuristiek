\documentclass{article}
\usepackage{apacite} 
\usepackage[dutch]{babel}
\usepackage{graphicx}
\usepackage{float}
\usepackage{wrapfig}
\begin{document} 


\begin{titlepage}

\center
\huge{ \bfseries Social media als communicatiemiddel tussen klant en bedrijf.}%titel
\\~
\\~

\begin{minipage}{0.4\textwidth}
\begin{flushleft} \large
\emph{Auteur:}\\
Nienke Pot
\end{flushleft}
\end{minipage}
~
\begin{minipage}{0.4\textwidth}
\begin{flushright} \large
\emph{Supervisor:} \\
Dick Heinhuis
\end{flushright}
\end{minipage}\\[2cm]

{\large \today}\\[2cm]	%today's date

\includegraphics[width=0.4\textwidth]{logo.png}\\
\end{titlepage}

\begin{newpage}
\begin{abstract}
Your abstract.
\end{abstract}
\end{newpage} 
 
\begin{newpage}
 \tableofcontents
\end{newpage} 

 \begin{newpage}
 
 \section{Inleiding} %mooi verhaal van maken
Generatie Y staat voor een generatie die hun hele leven is opgegroeid in een digitale omgeving(geboren 1981-1999). De informatie technologie heeft een groot effect op hoe deze generatie leeft en werkt. De meesten van generatie Y zijn opgegroeid met de computer, ze weten hoe en waarvoor je deze moet gebruiken\cite{aksoy2013understanding}. 


\citeA{aksoy2013understanding} stelt dat, de noodzaak om met anderen te willen communiceren de reden is dat generatie Y gebruik maakt van social media. 

Deze generatie maakt ook liever gebruik van social media om te communiceren met vrienden en familie dan dat oudere generaties dat doen. Dit heeft ook te maken met dat deze generatie hier meer ervaring mee heeft. Het gebruik van social media door generatie Y heeft invloed op de verwachtingen die mensen hebben op de service. Dit heeft invloed op hoe organisaties hun communicatiekanalen inrichten \cite{aksoy2013understanding}. 

Steeds meer vervangen bedrijven hun traditionele communicatiekanalen voor nieuwe communicatiekanalen \cite{tinnila1995model}.
De nieuwe kanalen zijn veel interactiever dan de traditionele kanalen \cite{klaus2013case} en sluiten beter aan op de behoeftes van generatie Y.  Maar weinig bedrijven hebben verstand van hoe ze deze nieuwe communicatiekanalen moeten toepassen  \cite{tinnila1995model}.

Het is wel van belang dat de juiste communicatiemiddelen worden ingezet om de relatie tussen generatie Y als klant en een bedrijf goed te houden.  Communicatie is bij een relatie van groot belang om de klant aan te trekken en te behouden. Het communicatiekanaal moet toegankelijk zijn voor de klant, zodat het de klant niet wegdrijft.

Relaties tussen bedrijf en klant komen voor op verschillende manieren afhankelijk van de type service die een bedrijf levert. Elk type service vraagt om een andere strategie in het contact met de klant \cite{silvestro1992towards}. Het contact met de klant kan verschillen in de duur van het contact en in de intimiteit \cite{raciti2010embedding}. 

De keuze voor het communicatiemiddel kan dus per type service dat een bedrijf levert verschillen. Ook is het belangrijk dat bij de keuze voor een communicatiemiddel wordt gekeken naar de behoefte van de klant. Generatie Y vindt het prettig om via social media te communiceren, dus voor bedrijven is het van belang om dit voor deze klanten te implementeren.  Maar voor welke social media moeten bedrijven kiezen?

 \subsection{Probleemstelling}
In de literatuur wordt weinig gezegd over wanneer welke vormen van social media geschikt zijn als communicatiemiddel. Het is wel van belang voor een bedrijf om te kijken naar welk communicatiemiddel voor hen het meest geschikt is afhankelijk van de type service die ze bieden. Het doel is om de relatie met de klant zo hecht mogelijk te maken. De communicatie via het kanaal moet dus voor de klant als zo prettig mogelijk ervaren worden. De opkomst van social media heeft voor andere verwachtingen in de communicatie bij de klant gezorgd. En bedrijven zien steeds meer het belang in om social media als klantkanaal te gebruiken. Maar welke social media is nu het meest geschikt voor een bedrijf, elk bedrijf levert namelijk weer een ander type service.  Dit probleem leidt tot de hoofdvraag van dit paper. \\
 
{\bf Hoofdvraag:}
In hoeverre is er een verschil in het gebruik van social media als communicatiemiddel per type service dat een bedrijf levert?\\
\\
De hoofdvraag wordt beantwoord aan de hand van drie deelvragen.  \\

{\bf Deelvraag 1:}
Welke factoren zijn van invloed op de keuze van de klant voor een social media kanaal per type service van het bedrijf?\\

{\bf Deelvraag 2:}
Is er een verschil in keuze van de klant voor een social media kanaal om met een bedrijf te communiceren afhankelijk van het type service van het bedrijf?\\

{\bf Deelvraag 3}
Is het verschil in keuze van de klant voor een social media als kanaal om met een bedrijf te communiceren afhankelijk van het type service van het bedrijf empirisch te bewijzen?\\

De eerste twee deelvragen zullen worden beantwoord in het theoretisch kader.
De eerste deelvraag wordt beantwoord aan de hand literatuuronderzoek. Nadat deze is  beantwoord, zal er aan de hand van literatuuronderzoek een model worden opgesteld om antwoord te geven op de tweede deelvraag. Dit model wordt vervolgens in een onderzoek getoetst en dan wordt antwoord gegeven op de derde deelvraag. Als de drie deelvragen zijn beantwoord kan in de conclusie antwoord worden gegeven op de hoofdvraag. 

\subsection{Relevantie}
Het doel van dit onderzoek is om te onderzoeken voor welk service type welke type social media het meest geschikt is. Bedrijven maken veelal gebruik van social media om contact te hebben met de klant naast de andere bestaande kanalen. Nog steeds is er weinig bekend over hoe mensen interacteren met het internet \cite{silvestro1992towards}. Er is echter weinig literatuur die zich specifiek richt op social media als klantkanaal, terwijl het juist belangrijk is voor bedrijven om dit medium goed te gebruiken. Social media groeien elk jaar steeds meer en raken steeds meer verweven in het leven van de klant en daar moeten ze goed op inspelen. Dit kan alleen als je een goed idee hebt over welk social media je wanneer moet toepassen en niet door gewoon maar iets te doen. Dit onderzoek laat zien, vanuit het perspectief van de klant, wat de klant als meest prettig ervaart. Die klantervaring heeft grote invloed op de relatie met die klant.

\subsection{ Scope en definities}
Er is veel literatuur over het communiceren met klant en het beheren van de kanalen waarover klant en bedrijf communiceren. Het onderwerp is erg breed en er zijn veel mogelijkheden om naar te kijken. In deze scriptie is gekozen om te kijken vanuit het perspectief van de klant. Het gaat hier om de factoren die een klant be\"invloeden om gebruik te maken van een bepaald social media om met een bedrijf te communiceren.  Aan de ene kant is er het proces die bepaald hoe een klant voor een bepaald kanaal kiest om mee te communiceren. Aan de andere kant moet worden bekeken wat de invloeden zijn bij geschreven communicatie, omdat social media, in dit geval Facebook en Twitter, voornamelijk communicatie door tekst is. 
  
Communicatie met een klant kan op veel manieren. Een klant kan een klacht, vraag, verzoek hebben, het bedrijf kan reclame maken en zo zijn er nog veel meer manieren om te communiceren. Omdat in dit onderzoek niet alle . In dit onderzoek gaat het om klanten die vragen stellen aan het bedrijf. Deze manier van communiceren is interessanter, omdat deze manier van communiceren uit twee richtingen komt en het bedrijf afhankelijk van het type ook anders zal reageren.
 
 \subsubsection{Social Media}
 In dit onderzoek wordt de term social media veel gebruikt. De definitie van social media die in dit onderzoek wordt gebruikt, wordt hieronder beschreven.
 Volgens \citeA{kaplan2010users} kan social media als volgt worden gedefineerd:
\begin{quote}
"Social Media is a group of Internet-based applications that build on the ideological and technological foundations of Web 2.0, and that allow the creation and exchange of User Generated Content."
\end{quote}
Social media is een online dienst waar gebruikers inhoud kunnen cre\"eren en delen. Ook kunnen gebruikers van social media met elkaar communiceren en bouwen ze een band op met elkaar. Social media bestaat al sinds 1981, maar vanaf 2003 is het echt in gebruik. 

 In het onderzoek worden alleen de twee social media Twitter en Facebook met elkaar vergeleken. De vijf grootste Social Media platforms in Nederland zijn;
 1. Facebook, 2. Youtube, 3. Google+, 4. LinkedIn, 5. Twitter  \cite{NSMO2015}.\\
\begin{figure}[H]
    \centering
    \includegraphics[width=0.3\textwidth]{SM5.png}
    \caption{Voorkeur klantkanaal}
    \label{fig:Voorkeur klantkanaal}    
\end{figure}

Niet alle vijf de social media zijn even geschikt om als klantkanaal te gebruiken. 
Youtube is voornamelijk een platform voor het delen van filmpjes. Het wordt wel gebruikt door bedrijven om zichzelf te presenteren, maar niet om te communiceren met de klant.  LinkedIn wordt voornamelijk gebruikt als zakelijk platform en is niet gemaakt voor het onderhouden van klantcontacten. Daarom valt dit medium af. Google+ is een media dat in Nederland sinds 2013 opkomt en vrij nieuw is. De vele leden die het medium bevat komt voornamelijk, omdat iedereen met een google account ook automatisch een Google+ account heeft. Het aantal gebruikers dat actief is op het medium is niet altijd even duidelijk omdat bijvoorbeeld ook reacties op YouTube worden meegerekend. Omdat Google+ nog zo opkomend is en nog niet heel duidelijk is waarvoor het allemaal wordt gebruikt, is de reden dat dit medium niet mee wordt genomen in het onderzoek.
Twitter en Facebook zijn platforms waarop bedrijven zich duidelijk representeren en waarop veel gecommuniceerd wordt tussen gebruikers. Daarom is gekozen om alleen Twitter en Facebook met elkaar te vergelijken. 

  \subsubsection{Service type}
  In dit onderzoek wordt gezocht naar verschillende communicatiekanalen per type service. Hieronder wordt beschreven welke type services worden onderscheden in dit onderzoek.
 Er zijn 6 dimensies waar service types op kunnen  worden onderscheden \cite{silvestro1992towards}. \\

 {\em Materiaal/Mens focus}
 
Bij een bedrijf dat materiaal als focus heeft is de voorziening van een bepaald materiaal het belangrijkste onderdeel van de service.  Als het bedrijf de focus op de mens heeft liggen, is het belangrijkste deel van de service de voorziening met een contact persoon.\\
 
 {\em Lengte tijd van klantcontact}
 
Bij een hoge klantcontact heeft de klant uren, dagen of weken contact met het bedrijf. Bij lage klantcontact heeft de klant maar een paar minuten contact met het bedrijf.\\
 
{\em Mate van maatwerk}
 
Bij een hoge maat van maatwerk wordt de service proces aangepast naar de behoefte van de klant. Bij een lage maat van maatwerk is er gestandaardiseerd service proces, de klant heeft vaak keuze uit een aantal opties die al vaststaan.\\
 
 {\em Mate van discretie} 
 
Als de mate van discretie hoog is dan mogen frontoffice medewerkers zelf meer beslissingen maken over aanpassingen van de service. Bij een lage mate van discretie moeten frontoffice medewerkers aan hun supervisor toestemming vragen voor aanpassingen aan de service.\\
 
{\em Bron van de toegevoegde waarde, frontoffice of backoffice}

Als een bedrijf meer backoffice ge\"orienteerd is, dan is het aantal frontoffice medewerkers weinig ten opzichte van het totaal aantal medewerkers.\\

 {\em Product/proces focus}

Als het bedrijf product ge\"orienteerd is dan draait het om wat de klant koopt. Is het bedrijf meer proces gefocust dan ligt de nadruk op de manier waarop een service aan de klant wordt geleverd \cite{silvestro1992towards} \cite{tinnila1995model}. \\

De zes dimensies correleren met het aantal klanten dat per dag wordt verwerkt. Als het aantal klanten toeneemt  dat door een unit wordt verwerkt per dag zijn de volgende trends te zien bij de zes dimensies:
\begin{itemize}
\item De focus schuift van ori\"entatie op mens naar materiaal.
\item De lengte van het contact met de klant wordt korter.
\item Mate van maatwerk wordt minder.
\item De mate van discretie wordt lager.
\item Het bedrijf schuift van frontoffice geori\"enteerd naar backoffice geori\"enteerd.
\item Focus schuift van proces geori\"enteerd naar product geori\"enteerd \cite{silvestro1992towards} \cite{tinnila1995model}.
\end{itemize}

Er zijn drie type services om te onderscheiden: %clusteren uitleggen ofzo
professionele services, service shop en massa services \cite{tinnila1995model} \cite{silvestro1992towards}\cite{ng2007typology}\cite{verma2000empirical}. Elke service type is gekarakteriseerd op basis van de 6 dimensies. Op sommige punten overlappen de categorie�n\cite{silvestro1992towards}.\\

{\em Professionele services}

Deze type organisatie heeft relatief weinig transacties, met een hoge mate van maatwerk en lange klantcontacten. Het bedrijf is proces en frontoffice geori\"enteerd \cite{ silvestro1992towards} \cite{verma2000empirical}. Er is een hoge maat van discretie om aan de behoefte van de klant te voldoen\cite{silvestro1992towards}.\\

{\em Service shop} 

Deze type valt tussen de andere twee categorie\"en in. Ook de classificaties van de dimensies valt tussen de twee categorie\"en in \cite{silvestro1992towards} . \\

{\em Massa services} 

De type organisatie heeft veel transacties, met een lage maat van maatwerk en een kort klantcontact \cite{silvestro1992towards} \cite{verma2000empirical}.
Het bedrijf is product en backoffice geori\"enteerd. De frontoffice medewerkers hebben een lage maat van discretie\cite{silvestro1992towards}.\\

\begin{figure}[H]
     \centering
    \includegraphics[width=0.8\textwidth]{stypen.png}
    \caption{Service typen}
    \label{fig:Service typen}    
\end{figure}
   
 \section{Theoretisch kader}
In het literatuuronderzoek zijn alleen de top 5 journals op het gebied van marketing service \cite{svensson2008scientific}  en de top 10 op het gebied van library and information science gebruikt \cite{aharony2012library}. \\

\begin{table}[htdp]
\caption{De top 5 journals op het gebied van marketing en service}
\begin{center}
\begin{tabular}{|c|c|}
	\hline
	International Journal of Service Industry Management \\
	Journal of Services Marketing \\
	Journal of Service Research \\
	Managing Service Quality \\
	Service Industries Journal \\
	\hline
\end{tabular}
\end{center}
\label{default}
\end{table}%

\begin{table}[htdp]
\caption{De top 10 journals op het gebied van library and information science}
\begin{center}
\begin{tabular}{|c|c|}
	\hline
	Journal of the American Society for Information Science and Technology\\
	Scientometrics\\
	Journal of Information Processing \& Management \\
	Journal of Computer-Mediated Communication\\
	Journal of Information Science\\
	Journal of Documentation\\
	Information Research\\
	College \& Research Libraries\\
	Library \& Information Science Research \\
	Journal of Global Information Management \\
	\hline
\end{tabular}
\end{center}
\label{default}
\end{table}%

De top 5  journals op het gebied van marketing en service is vastgesteld door \citeA{svensson2008scientific} zij bepaalde deze top journals op basis van hun wetenschappelijke identiteit. De top 5 is in hun onderzoek vastgesteld door tien onafhankelijke wetenschappers uit Noord-Amerika, Europa en Australi\"e. De top 10 journals op het gebied van library and information science is bepaald door \citeA{aharony2012library}. De top tien is vastgesteld op basis van hun impactfactor op het gebied van library and information science. 

%vertellen wat je gaat vertellen waarom deze theorien.
%inleidende stukjes

\subsection{Keuze klantkanaal}
De eerste stap die een klant maakt is de keuze voor het klantkanaal. Daarom moeten bedrijven inzicht krijgen in het proces dat zich afspeelt als de klant een keuze maakt voor een klantkanaal. Vervolgens moeten bedrijven de klantkanalen integreren in het bedrijf.
  
 \citeA{neslin2006challenges} stelt dat er vijf grote uitdagingen zijn voor het kiezen, evalueren en implementeren van klantkanalen voor bedrijven; data integratie, begrijpen van klantgedrag,evaluatie van het klantkanaal, verdeling van middelen langs de verschillende klantkanalen en co"ordinatie van de klantkanaal strategie.  
De eerste uitdaging is {\em data integratie}. De data integratie bekijkt welk kanaal de klant gebruikt tijdens de verschillende fases van het beslissingsprocess. Daarbij moet ook gekeken worden bij de concurrenten. De focus kan liggen op het zoeken naar een ideaal kanaal of kan kijken hoe je het gebruik van het bestaande kanaal kan optimaliseren 
De tweede uitdaging is het {\em begrijpen van klantgedrag}. Het is belangrijk voor managers om inzicht te hebben op de manier waarop klanten hun klantkanaal kiezen en wat de invloed hiervan is op het koopgedrag van de klant.
De derde uitdaging voor een bedrijf is de {\em evaluatie van het klantkanaal}. Als er inzicht is in de verzamelde data, kan worden gekeken hoe een klantkanaal presteert.
De vierde uitdaging is de {\em verdeling van middelen langs de verschillende klantkanalen}. Er moet worden besloten welke klantkanalen worden ingezet en op welke wijze.
De vijfde en laatste uitdaging is de {\em co"ordinatie van de klantkanaal strategie}. Om een samenhangend en samenwerkend geheel van de verschillende klantkanalen te cre\"eren, moeten de managers de doelen, het ontwerp en de ontwikkeling van de kanalen co\"ordineren\cite{neslin2006challenges}.
De vijf uitdagingen hangen als volgt samen; Door data integratie krijgen managers inzicht in het  klantgedrag en kunnen ze de prestatie van het klantkanaal evalueren. Daaruit volgt dat de strategie wordt bepaald, een belangrijk deel hiervan is de verdeling en co"ordinatie van de kanalen.

 \citeA{florenthal2010four} stellen dat interactie  de mate is waarin een klant handelt of reageert op een bron. Deze bron kan worden opgesplitst in vier categorie\"en; mens, medium, bericht en product. Dit zijn de vier mogelijkheden van bronnen waarmee een klant interactie mee heeft. Met deze 4 categorie\"en kunnen online en offline klantkanalen worden ge\"evalueerd en met elkaar worden vergeleken. 
De categorie {\em mens} onderscheidt zich van de andere categorie\"en door de aanwezigheid van een persoon tijdens de interactie. Het voordeel van deze categorie is dat bij een eenmalige ontmoeting met een vreemde een innige relatie makkelijker wordt gevormd door de aanwezigheid van beide partijen. 
De volgende categorie {\em bericht} richt zich op het vermogen van de mens om de inhoud en presentatie van een bericht aan te passen naar de behoefte van de klant. Deze categorie richt zich erg op het leveren van maatwerk. Hierbij worden de offline en online klantkanalen onderscheiden op het gebied van interactie tussen mens en computer. 
De derde categorie {\em medium} kijkt naar de relatie tussen mens en interface oftewel de ervaring met virtuele omgeving. Deze interactie kan door een goede toegankelijkheid en navigatie worden versterkt. Bij toegankelijkheid gaat het erom dat de klant het klantkanaal snel, makkelijk en op elk moment vanaf elke locatie kan gebruiken. Bij een goede navigatie gaat het erom dat de klant snel kan vinden wat hij zoekt.
De laatste categorie {\em product} focust zich op hoe mensen met objecten interacteren en het vermogen om acties op deze objecten uit te voeren. De type acties die mogelijk zijn uit te voeren bij een klantkanaal zijn belangrijker dan de aantal acties die uitgevoerd kunnen worden. Wel geldt hoe meer acties worden waargenomen, des te meer interactief een klantkanaal wordt ondervonden \cite{florenthal2010four}.

\begin{figure}[h]
     \centering
    \includegraphics[width=0.8\textwidth]{voorkeur.png}
    \caption{Voorkeur klantkanaal}
    \label{fig:Voorkeur klantkanaal}    
\end{figure}

Het is voor het kiezen van klantkanalen belangrijk om te kijken naar welk kanaal de klant gebruikt voor welk doel en welke actie hieruit plaats vindt. Op basis van de verwachting en voorkeur van een klant voor een van de vier categorie\"en van bronnen kiest hij een klantkanaal\cite{neslin2006challenges} \cite{florenthal2010four}\cite{javalgi2006marketing}.
De ervaring wordt als pretting ervaren als de verwachtingen van de klant overeenkomen met de ervaring met de klantkanaal, als de verwachting niet overeenstemt met de ervaring dan wordt dit als niet prettig ervaart door de klant \cite{florenthal2010four}\cite{javalgi2006marketing}. De klant leert en evalueert de voorafgaande ervaring met klantkanalen, dit geeft feedback op de verwachting en voorkeur van de klant voor het kiezen van een klantkanaal \cite{neslin2006challenges}( Figuur 3).%checken of dit klopt.

\subsection{Klantervaring}
De klant gebruikt zijn ervaringen met het klantkanaal als hij een nieuw klantkanaal kiest. Het is daarom belangrijk om te weten van welke factoren de klantervaring vanaf hangt.

Hoe een klant de online service ervaart is te bepalen aan de hand van twee dimensies die zijn opgedeeld in acht sub-dimensies. De twee dimensies zijn functionaliteit en psychologische factoren. De attributen van de dimensies kunnen zowel een negatieve als positieve invloed hebben op hoe de klant de online dienst ervaart en tot hoe een klant zich aangetrokken voelt tot de online dienst \cite{klaus2013case}.

De sub-dimensies van de psychologische dimensie hebben betrekking op het weghalen van de weerstand van de klant om een online dienst te gebruiken. Deze weerstand is er vaak, omdat de klant het merk of kanaal niet goed kent en de fysieke aanwezigheid van de site ontbreekt\cite{klaus2013case}. 

{\em Vertrouwen} in de online dienst is voor de klant erg belangrijk. De fysieke afstand, gebrek aan persoonlijk contact en de anonimiteit op het internet zorgt ervoor dat mensen huivering zijn voor het gebruik ervan. Het vertrouwen winnen van de klant is daarom erg belangrijk.
De {\em waarde voor geld} voor een product of service wordt online als goedkoper ontvangen. En dat is waar de klanten veel waarde aan hechten.
{\em Herkenbaarheid van de context} wordt gecre�erd als de visuele representatie van de website consistent is\cite{klaus2013case}.

De sub-dimensies van de functionele dimensie hebben betrekking op de technische prestatie van de online dienst \cite{klaus2013case}.

De {\em  gebruiksvriendelijkheid} is gerelateerd aan de  attributen die ervoor zorgen dat de klant het gebruik van de online dienst als aangenaam ervaart.  De klant heeft duidelijke verwachtingen dat de dienst ze moet helpen hun doel te bereiken.
Als de klant de dienst als snel en makkelijk ervaart, zijn ze geneigd om opnieuw gebruik te maken van de online dienst. Als de dienst niet aan hun verwachtingen voldoet zal de klant de dienst niet opnieuw gebruiken.
De {\em  communicatie} draait om attributen die klanten een veilig gevoel geven bij de online dienst. Bijvoorbeeld het krijgen van een bevestigingsmail geeft de klanten een veilig gevoel en voor hun gevoel vermindert dat de risico van oplichting bij de online dienst.
De {\em  aanwezigheid van het product} geeft de klant het gevoel dat hij in een winkel is en stimuleert  een aankoop. Het is daarom een vereiste dat het product aanwezig is en er moet de mogelijkheid zijn om met het product interactie te hebben; het klikken om extra informatie te krijgen of een nog meer plaatjes ervan zien.
Door {\em  interactiviteit} kan de service meer gepersonaliseerd worden en kan er meer maatwerk worden geleverd. Een voorbeeld is dat een online dienst producten kan aanraden aan de hand van de zoek of koopgeschiedenis van de klant.
{\em   Sociale aanwezigheid} gaat over de interactie met andere klanten. Deze interactie verloopt via commentaren en beoordelingen. Het lezen van de ervaring van andere klanten zorgt ervoor dat de klant minder huivering is voor de onbekende online dienst en die fysieke afwezigheid van de dienst. De ervaringen van de andere kanten kunnen klanten overhalen een dienst te gebruiken of juist niet \cite{klaus2013case}.


\subsection{Communicatie door text}
De communicatie op social media verloopt voornamelijk via het versturen van tekst. Daarom is het belangrijk voor bedrijven om te begrijpen hoe communiceren met de klant via tekst zo goed mogelijk verloopt, zodat de klant tevreden is. 
De communicatie tussen bedrijf en klant speelt een belangrijke rol in het onderhouden van de klantrelatie, geschreven communicatie is een groot onderdeel hiervan. De bedoeling van communicatie voor een bedrijf is het zorgen van een hechtere relatie met de klant \cite{raciti2010embedding}.

Communicatie is het overbrengen van een boodschap naar anderen op een manier dat het door de andere partij begrepen wordt en zinvol is\cite{raciti2010embedding}. Bij geschreven communicatie zijn er 2 processen betrokken. Het eerste proces is het cre\"eren van een tekst, daarop volgt het tweede proces de interpretatie van de tekst door de ontvanger. Op de interpretatie van de ontvanger kan een nieuwe creatie van tekst ontstaan en dan begint het proces opnieuw, dit hoeft echter niet altijd zo te zijn \cite{dickey2006you}. De gecre\"erde tekst brengt een boodschap over en deze wordt ook vastgelegd. In tegenstelling tot mondelinge communicatie kan naar geschreven communicatie altijd weer worden terugverwezen, omdat dit ergens is vastgelegd. De verzonden tekst leidt pas tot communicatie als het wordt gelezen en ge\"interpreteerd. Bij het interpeteren maken we gebruik van ons verstand om de tekst te reconstrueren tot iets wat betekenis heeft, dit process hangt af van de ervaringen die een persoon heeft. Het is belangrijk om gemeenschappelijke opvattingen te hebben over de betekenis van de tekst om miscommunicatie te voorkomen \cite{dickey2006you}.

De beleving van de communicatie door de klant is belangrijk, een goede beleving kan een relatie hechter maken of een slechte beleving kan de relatie slechter maken. Er zijn vier indicatoren die belangrijk zijn in de geschreven communicatie, omdat zij van invloed zijn op hoe de klant de relatie met het bedrijf beleeft\cite{raciti2010embedding}. 
De belangrijkste indicator is {\em helderheid}. Een tekst is helder als er een logische structuur in zit,  duidelijke zinnen bevat, geen herhaling heeft en er geen niet uitgelegde woorden in zitten. Ook bevat de tekst alleen relevante informatie en heeft een warme toon. Een heldere tekst zorgt voor vertrouwen bij de klant \cite{raciti2010embedding}.
Daarna is {\em esthetiek} de meest belangrijke indicator. De esthetiek van een tekst volgt uit de presentatie van een tekst door de lettertype en grootte. Een goede presentatie van een tekst maakt de tekst makkelijker om te lezen \cite{raciti2010embedding}.
Daarop volgt de indicator {\em nauwkeurigheid}. Deze indicator duidt op de nauwkeurigheid van de geschreven communicatie. Het gaat hier om correcte grammatica, spelling en interpunctie. Deze nauwkeurig laat zien dat de focus op de klant ligt, de klant voelt zich hierdoor belangrijk \cite{raciti2010embedding}.
De indicator die het minst van invloed is {\em fysieke kenmerken}\cite{raciti2010embedding}. De fysieke kenmerken op social media zijn de vormgeving van een bericht zijn, achtergrondkleur en lay-out.

\begin{figure}[H]
     \centering
    \includegraphics[width=0.9\textwidth]{tekstcom.png}
    \caption{Communicatie door tekst}
    \label{fig:Communicatie door tekst}    
\end{figure}

\subsection{Keuze voor social media}
Door het model van de communicatie door tekst(Figuur?) en de keuze van klantkanaal(Figuur?) te combineren kom je op het model voor de keuze van het social media kanaal(Figuur?). Dit model beschrijft hoe de klant de keuze maakt voor het gebruik van een social media kanaal om in contact te komen met een bedrijf.

De klant heeft een vraag voor een bedrijf. Aan de hand van de vraag heeft de klant ook een bepaalde verwachting van via welk social media hij het beste antwoord kan verwachten. Uiteraard heeft de klant ook een voorkeur voor een bepaald kanaal, omdat hij daar bijvoorbeeld beter bekend mee is.  Aan de hand van de voorkeur en de verwachting van de klant kiest de klant een kanaal om zijn vraag over te sturen. De manier waarop de klant de vraag formuleert, de 4 indicatoren, heeft vervolgens invloed op hoe het bedrijf de vraag vervolgens interpreteert en daarop een antwoord weer terug stuurt over het zelfde kanaal naar de klant. De manier waarop het bedrijf het antwoord weer vervolgens formuleert is weer van invloed op hoe de klant het antwoord interpreteert. Vooral voor het bedrijf is het belangrijk dat het antwoord bij de klant opdezelfde manier overkomt als die door hen bedoelt is. Als de boodschap voor beide hetzelfde is dan is er goede communicatie wat een positief effect heeft op de klantrelatie. Na het antwoord van het bedrijf evalueert de klant hoe de communicatie verlopen is en past daarop zijn voorkeur en verwachting eventueel aan. 

\begin{figure}[H]
     \centering
    \includegraphics[width=1.3\textwidth]{smkanaal.png}
    \caption{Keuze voor social media kanaal}
    \label{fig:Keuze voor social media kanaal}    
\end{figure}

\subsection{Hypotheses}
Drie type services zijn vastgesteld om te onderscheiden; professionele services, service shops en massa services. Voor elke van de drie services is een hypothese gesteld over welk type social media, Twitter of Facebook, het meest geschikt is. De voorkeuren zijn bepaald aan de hand van het model van de keuze voor het social media kanaal(Figuur?)\\

{\em H1: Voor professionele services is Facebook meer geschikt als klantkanaal dan Twitter.}


{\em H2: Voor service shops is Facebook meer geschikt als klantkanaal dan Twitter.}

{\em H3: Voor massa services is Twitter meer geschikt als klantkanaal dan Facebook.}
De klant verwacht een kort antwoord van een massa service, maar de klant wil het liefst wel snel een antwoord. Twitter is meer real-time dan Facebook, de antwoorden zijn over het algemeen snel en kort en daarom meer geschikt voor massa services. Door het gebruik van Hashtags en links zijn de antwoorden wel korter, maar wel helder. De nauwkeurigheid ligt hier lager omdat daar minder woorden voor zijn, maar massa services zijn ook minder persoonlijk dus wordt dat ook niet verwacht door de klant.

\section{Methode}
Om vast te stellen of empirisch bewezen kon worden of het model over de keuze van social media kanaal(Figuur 5) klopt, is er een enqu\^ete afgenomen. Deze enqu\^ete heeft vastgesteld of de hypotheses kloppen. Er worden twee groepen getoetst. Groep A voor Facebook en groep B voor Twitter. 

Elke groep kreeg 3 simulaties te zien. Dit was afbeelding van een gesprek tussen klant en bedrijf via social media (Facebook of Twitter). In elke simulatie werd een bedrijf gekozen dat een van de drie type services representeerde. Een fictief architecten bureau representeerde de professionele services, een fictief hotel voor de service shops en de ANWB voor de massa services. Bij het kiezen van de bedrijven die de services representeerde is ook nagedacht over wat voor houding klanten eventueel tegenover die bedrijven hebben. Bijvoorbeeld de belastingdienst is een massa service, maar veel klanten hebben een negatieve associatie bij deze organisatie. Dan is het niet eerlijk om deze te vergelijken met een hotel, waar klanten eerder een positieve associatie bij hebben. 

Nadat de respondenten de simulatie gezien hadden, werden ze gevraagd of ze het antwoord van het bedrijf konden herhalen, om te kijken of de boodschap van het bedrijf inderdaad goed over was gekomen. Daarna werden de respondenten vragen gesteld over de effectiviteit van het bericht. De effectiviteit hangt hier af van de nut, vertrouwen, loyaliteit en makkelijkheid. 

Deze vier factoren worden ook gebruik in het onderzoek van ....?. 

De vragen konden worden beantwoord op de schaal van Likert van totaal niet tot heel erg.

De schaal van Likert is in dit onderzoek als ordinaal beschouwt. uitleg...

\section{Resultaten}

\section{Conclusie}

\section{Discussie}

\section{Limitaties}






 \end{newpage}
\begin{newpage}
 
\bibliographystyle{apacite}
\bibliography{aharony2012library.bib,aksoy2013understanding.bib,hennig2010impact.bib,svensson2008scientific.bib,kim2012frequency.bib,neslin2006challenges.bib,NSMO2015.bib,florenthal2010four.bib,kaplan2010users.bib,javalgi2006marketing.bib,raciti2010embedding.bib,dickey2006you.bib,silvestro1992towards.bib,ng2007typology.bib,verma2000empirical.bib,tinnila1995mode.bib,klaus2013case.bib}
 
\end{newpage}
\appendix
\section{Appendix}
\section{iets}

\end{document}


   
   